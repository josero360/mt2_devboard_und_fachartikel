\section{Mögliche Erweiterung}

Wenn für $U1$ ein Operationsverstärker mit Disable/Shutdown Pin verwendet wird, können mehrere Lasten mit 
verschiedenen $I_{L,max}$ parallel geschaltet werden. 
Es wäre nur eine Referenz nötig. 
So kann in einem Gerät eine breitbandige genaue 
(nach Abschnitt \ref{sec:Grenzen}) Stromsenke realisiert werden. 
Beispiel: $I_{L1,max} = \SI{100}{m\ampere}, I_{L2,max} = \SI{1}{\ampere}, I_{L3,max} = \SI{10}{\ampere}$.\\

Wenn nun mehrere Spannungsreferenzen zur Verfügung stehen, können nach dem Kirchhoffschen-Knotenpunktsatz auch \grqq Mischgrößen\grqq{} 
gebaut werden.
Beispiel: $I_{L,max} = \SI{1,2}{\ampere}$ ($\SI{1}{\ampere}$ und $\SI{200}{m\ampere}$)\\


Wie schon im Artikel angedeutet, wäre eine Erweiterungsmöglichkeit, andere Fehlerarten der Schaltung 
mit einem Mikrocontroller zu verhindern.
Informationswerte, wie aktueller Stromfluss und Spannung, können an einem Display angezeigt werden.
So könnte beispielsweise eine Temperaturüberwachung am Kühlkörper realisiert werden.
Bei überschreiten der Grenztemperatur kann der Shutdown des OP`s genutzt werden, um den MOSFET hochohmig zu schalten.
Mit einem weiteren ADC Eingang, über einen Spannungsteiler gemessen, kann $P_{Q1} = (U_{L} - R1 \cdot I_{L}) \cdot I_{L}$ bzw. 
$P_{L} = U_{L} \cdot I_{L}$ berechnet werden. 
Ein akustisches oder visuelles Signal kann so vor den bevorstehenden Überschreitungen von 
$U_{DS,Q1}$, $I_{D,Q1}$ und/oder $P_{tot,Q1}$, den Nutzer warnen.\\

Wer ein passendes Analogmessinstrument hat, kann die Anzeige auch im Retrostil bauen.