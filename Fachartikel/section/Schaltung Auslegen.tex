\section{Schaltung Auslegen}

Falls Sie die Schaltung zusammenbauen möchten sollte der nachfolgende Teil einmal vollständig gelesen werden.
Sie legen Ihre Präferenzen fest, berechnen die Werte und suchen dann passende Bauteile heraus.
\newline

Wenn der MOSFET ($Q1$) im Betrieb hochohmig geschaltet werden soll, wird für die Regelung ($U1$), 
ein sogenannter Rail to Rail Operationsverstärker benötigt. 
Diese Art von OP`s, können ihre Ausgangsspannung bis nah an die Versorgungsspannung heran durchschalten.

Bei der Versorgung des OP`s muss darauf geachtet werden, dass das mögliche Ausgangspotential größer ist als $U_{GS(th),max}$.
$U_{GS(th),max}$ ist die maximale Gate-Source Schwellspannung (Englisch: Gate-Source Threshold Voltage) des MOSFET. 
Beim unterschreiten dieser, kann der MOSFET nicht vollständig durchschalten.
Bei einer Versorgungsspannung von \SI{5}{\volt} empfehlen sich sogenannte Logic-Level MOSFET`s 
(IRL(\textit{X}) \cite{LL_MOSFET}).

Neben diesem Wert sollte ebenfalls auf $U_{DS}$ (Drain-Source Durchbruch Spannung) geachtet werden. Diese bestimmt die 
spätere maximale Spannung der Last ($U_{L,max}$)\footnote{In Abb.\ref{sch:Grundschaltung} Spannung zwischen +Load und -Load}. 
Lesende, die mit dem Kirchhoffschen-Maschengesetz vertraut sind werden sagen, dass das noch nicht ganz stimmt.
Die Masche für die maximale Spannung ist:
\begin{equation}
	U_{L,max} = (I_{L} \cdot R1) + U_{DS}
	\label{eq:U_Lmax}
\end{equation}

Wenn bei dem Einschalten von einem Laststrom $I_{L} = \SI{0}{\ampere}$ ausgegangen wird, wird der erste Term aus Formel \ref {eq:U_Lmax} 
$= 0$ und $U_{L,max}$ ist somit gleich $U_{DS}$. 


Kleine Anmerkung: Wer ein bisschen bastelt, hat bestimmt ein IRFP450 \cite{IRFP450} herumliegen. 
Der eignet sich sehr gut: $U_{GS(th),max} = \SI{4}{\volt}$, $U_{DS} = \SI{500}{\volt}$, $I_{D} = \SI{14}{\ampere}$ 
und $R_{DS(on)} = \SI{0,4}{\ohm}$ (Bedingung Beachten \cite{IRFP450}).

Der Shunt-Widerstand $R1$ sollte allgemein nicht zu klein ausfallen.
Bei kleiner Loop (1)- und Referenzspannung, können elektromagnetische Störungen von Außen überwiegen.

Die minimale ohmsche Last die simuliert werden kann, ist:
\begin{equation}
	R_{L,min} = R1 + R_{DS(on)}
	\label{eq:R_Lmin}
\end{equation}

%\begin{equation}
%	P_{R1,max} = I_{L,max}^2  \cdot R1
%	\label{eq:P_R1max}
%\end{equation} 

\subsection{Beispiel} 
\label{sec:Bsp}
Nehmen wir ein IRFP450 \cite{IRFP450}, $R1 = \SI{100}{m\ohm}$ und $U_{U1} = U_{Ref}= \SI{12}{\volt}$.
Wir legen $I_{L´,max} = \SI{7}{\ampere}$, $U_{L,max} = \SI{20}{\volt}$ und $U1$ als Rail to Rail Operationsverstärker fest.

Somit ergeben sich für $P_{R1,max}$ und $U_{U1+,max}$: 
\begin{equation}
	P_{R1,max} = I_{L,max}^2  \cdot R1 = (\SI{7}{\ampere})^2 \cdot \SI{0,1}{\ohm} = \SI{4,9}{\watt}
\end{equation}
\begin{multline}
	\begin{split}
		U_{U1+,max} &= I_{L´,max} \cdot R1 = \SI{7}{\ampere} \cdot \SI{0,1}{\ohm} = \SI{0,7}{\volt} \\
		&= U_{R1,max}
	\end{split} 
\end{multline}


Bei der Referenz legen wir $R3 = \SI{10}{k\ohm}$ und $R2 = \SI{100}{k\ohm}$ fest. Formel \ref{eq:UOp+} 
%($R23$ siehe Formel \ref{eq:R23}) 
wird nach $R4$ umgestellt:
\begin{multline}
	\begin{split}
		R4 &= \biggl( \frac{U_{Ref} \cdot \frac{R2 \cdot R3}{R2 + R3}}{U_{U1+,max}}\biggr) - \frac{R2 \cdot R3}{R2 + R3}\\
		&= \biggl( \frac{\SI{12}{\volt} \cdot \SI{9,09}{k\ohm}}{\SI{0,7}{\volt}}\biggr) - \SI{9,09}{k\ohm} = \SI{146,73}{k\ohm}
	\end{split}
\end{multline}

Einen Widerstand mit $R = \SI{146,73}{k\ohm}$ wird es wahrscheinlich nirgends zu kaufen geben. 
Nach der E 96-Reihe (Nach DIN 41426 \cite{E-Reihen}), ist der nächste nähere Wert $147 k\Omega$. 
Durch diesen anderen Wert ergibt sich $I_{L,max} = \SI{6,988}{\ampere}$.


\subsection{EMV-Aspekte} 
\label{sec:EMV}
Eine Leiterschleife, senkrecht von einem magnetischen zeitlich-veränderlichen Fluss durchströmt, induziert eine elektrische 
Spannung. 
Die induzierte elektrische Spannung ist von der Fläche und der  magnetischen Flussdichte abhängig. Um diesen Störeinfluss 
%(wie im Abschnitt \ref{sec:Schaltung} angesprochen) 
zu minimieren, sollte der Loop 1 (Abb. \ref{sch:Grundschaltung}) möglichst kurz und eine kleine Innenfläche vorweisen. 
Bei Anwendung in Bereichen mit großen Störeinflüssen sollte der Loop 1 zusätzlich geschirmt 
werden (Bsp: Innenleiterbahn oder $\mu$-Metall). 
%Um Störeinflüsse an der Referenz zu minimieren sollte ein Kondensator parallel zu $R2$ möglichst nah am  Operationsverstärker 
%eingebracht werden.
Weniger kritisch ist die Referenz am Eingang $U1_{+}$. 
Hier sollte dennoch ein Kondensator parallel zu $R2$, möglichst nah am Operationsverstärker, 
eingebracht und lange Leiterbahnen/Zuleitungen vermieden werden.

\subsection{Temperaturabhängigkeit/-eigenschaft}

Die Temperatureigenschaften von dem MOSFET $Q1$ werden nicht beachtet, da dieser gesteuert wird. Die Temperaturabhängigkeit ist 
wesentlich von der Referenzspannung $U_{U1+}$ und dem Shunt $R1$ abhängig. Um den maximalen Fehler zu berechnen ist der TCR-Wert 
(Temperature Coefficient of Resistances), in [PPM/°C], wichtig. 
Dieser beschreibt die Widerstandsänderung aufgrund von Temperaturänderung. 
Für Widerstände gilt: (\cite{TCR_rechnen}, Seite 23)

\begin{equation}
	\label{eq:TCR}
	\Delta R_{max} = \Delta R + (T_{R,max} - \SI{25}{\celsius}) \cdot TCR_{R} \cdot10^{-4}
\end{equation}

\begin{table}[h!]
	\centering
	\caption{Beschreibung zur Formel \ref{eq:TCR}}
	\label{tab:Beschreibung_TCR}
	\normalsize
	\begin{tabular}{l|l}
		Term & Beschreibung \\
		\hline
		$\Delta R_{max}$& Resultierende maximale Toleranz[\%]\\
		$\Delta R$		& Widerstandstoleranz [\%] \\
		$T_{R,max}$		& Max. Temperatur am/im Widerstand [°C]\\
		$TCR_{R}$		& TCR-Wert des Widerstandes [PPM/°C]\\
	\end{tabular}
\end{table}

Um den Temperatureinfluss zu minimieren, sollten Widerstände konstant/nah bei \SI{25}{\celsius} betrieben werden. 
Der Shunt-Widerstand wird im Betrieb gegebenenfalls mit viel Leistung belastet. 
Hierbei sollte für ausreichend Kühlung gesorgt 
und bei der Fehlerrechnung der maximale Fehler nach Formel \ref{eq:TCR} beachtet werden.
Das Gehäuse des MOSFET sollte groß gewählt werden, damit die gegebenenfalls hohe anfallende 
Leistung, mit geringem thermischen Widerstand an den Kühlkörper abgegeben werden kann.
Die Kühlkörperberechnung/Auslegung ist nicht Teil dieses Artikels. 
Unter Bedingungen\footnote{$P_{L} \approx \SI{72}{\watt}$, \SI{30}{\second} Anzeit, \SI{10}{\minute} Auszeit} 
genügt ein relativ kleiner Kühlkörper von $\approx \SI{96,2}{c\meter^3}$.
Dabei wir die anfallende Wärme primär im Material gespeichert. 
Die Abgabe der gespeicherten Energie erfolgt größtenteils während der Auszeit.\\

Die restliche Schaltung ist thermisch und räumlich von diesen zwei zu kühlenden Bauteilen zu trennen.
Im Allgemeinen hat ein Schaltung eine höhere Genauigkeit, wenn diese bei konstanter und 
geringer Temperatur betrieben wird.